
\subsection*{Infomax}

The second topic we will consider is the Infomax algorithm. This is a
approach to unmixing mixed data proposed in \cite{BellSejnowski1995}
as giving an explanation for how the brain might perform auditory
source separation. The problem is as follows: imagine you are in a
crowded room, in the classic telling, at a cocktail party. Lots of
people are talking, but if you concentrate on one voice you can
separate it from the overall mixture. We can experience the opposite
effect when meditating or sitting in thoughtful silence, for example,
at a Quaker meeting: we suddenly notice how loud distance noises are,
the noise of voices on the street, other people coughing and clearing
their throats, the sound of wind through nearby trees. All these are
sounds we automatically filter out when listening. This process of
picking out individual sources of sound from a mixture of sounds is
called auditory source separation, or, indeed, the cocktail party
problem.

The problem can be phrased like this, given the sources
$\textbf{s}(t)$ where $\textbf{s}$ is a vector over multiple
sources. Now we do source separation with only two recordings, one for
each ear; here we are just going to consider the simpler problem of
source separation when there are as many recordings are there are
sources, we also assume the mixing is linear and instantaneous, real mixing of auditory signals in a room will only have these to problems approximately. However, given these assumption we have
\begin{equation}
\mathbf{r}(t)=M\mathbf{s}(t)+\mathbf{\epsilon}(t)
\end{equation}
with $M$ a square mixing matrix and $\mathbf{\epsilon}$ is some
noise. The goal is to find the unknown source signals $\textbf{s}(t)$
from the known recordings $\textbf{r}(t)$. This means finding the,
also unknown, matrix $M$.


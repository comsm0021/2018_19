\documentclass[12pt]{article}
\usepackage{amsfonts, epsfig}
\usepackage[authoryear]{natbib}
\usepackage{graphicx}
\usepackage{fancyhdr}
\usepackage{slashbox}
\pagestyle{fancy}
\lfoot{\texttt{comsm0021.github.io}}
\lhead{Neural Information Processing - 1\_information\_theory\_I - Conor}
\rhead{\thepage}
\cfoot{}
\begin{document}

\section*{Worksheet: outline solutions} 

\subsection*{Q1 - marginal and conditional distributions}

Work out the marginal probability distributions and the $x=a$ conditional probability distribution $P(Y|X=a)$ for
\begin{center}
\begin{tabular}{c|cc}
\backslashbox{$Y$}{$X$}&$a$&$b$\\
\hline
1&$\frac{1}{3}$&$\frac{1}{6}$\\
2&0&$\frac{1}{4}$\\
3&$\frac{1}{8}$&$\frac{1}{8}$
\end{tabular}
\end{center}

\subsubsection*{Solution}


Calculating the marginal distribution just requires adding along the rows or columns, if we take $X$ to be the random variable going across, so ${\cal X}=\{a,b\}$ and $Y$ the random variable going down, so ${\cal Y}=\{1,2,3\}$ then the two marginal distributions are, for $X$
\begin{center}
\begin{tabular}{c|cc}
&$a$&$b$\\
\hline
&$\frac{11}{24}$&$\frac{13}{24}$
\end{tabular}
\end{center}
and for $Y$
\begin{center}
\begin{tabular}{c|ccc}
&$1$&$2$&$3$\\
\hline
&$\frac{1}{2}$&$\frac{1}{4}$&$\frac{1}{4}$
\end{tabular}
\end{center}
The conditional distribution, conditioned on $x=a$ is calculated using
$p(y|x)p(x)=p(x,y)$ so we divide the $p(x=a,y)$ column by
$p(x=a)=11/24$, hence
\begin{center}
\begin{tabular}{c|ccc}
&$1$&$2$&$3$\\
\hline
&$\frac{8}{11}$&$0$&$\frac{3}{11}$
\end{tabular}
\end{center}



\subsection*{Q2 - working out entropy}

A fair coin is flipped until the first head occurs. Let $X$ denote the
number of flips required.
\begin{enumerate}
\item Find the entropy $H(X)$ in bits. The following expressions may be useful:\begin{eqnarray}
\sum_{n=0}^{\infty} r^n&=&\frac{1}{1-r}\cr
\sum_{n=0}^{\infty}nr^n&=&\frac{r}{(1-r)^2}
\end{eqnarray}
\item A random variable $X$ is drawn according to this
  distribution. Find a sequence of yes-no questions of the form, \lq
  Is $X$ contained in the set $S$?\rq{}. Compare $H(X)$ to the
  expected number of questions required to determine $X$. For the most
  efficient sequence, that is the sequence the shortest expected
  number of questions, these two numbers will be the same. You will
  find that the most efficient sequence is very straightforward!
\end{enumerate}

\subsubsection*{Solution}

So, here the set of possible outcomes is ${\cal X}=\{1,2,3,\ldots\}$ and we need to start by working out $p_X{n}$ the chance of throwing $n$ flips before getting a head. To get $X=n$ you need to throw $n-1$ tails, this has probability $1/2^{n-1}$ followed by a head, which has probability $1/2$, hence
\begin{equation}
p_X(n)=\frac{1}{2^n}
\end{equation}
It is easy to check that 
\begin{equation}
\sum_{n=1}^{\infty}p_X(n)=\sum_{n=1}^{\infty}\frac{1}{2^n}=\sum_{n=0}^{\infty}\frac{1}{2^n}-1=1
\end{equation}

Now, to calculate the entropy, just use the formula:
\begin{eqnarray}
H(X)&=&-\sum_{n\in {\cal X}}p_X(x)\log{p_X(x)}\cr
    &=&-\sum_{n=1}^{\infty}\frac{1}{2^n}\log{2^{-n}}=\sum_{n=1}^\infty\frac{n}{2^n}\cr
&=&2
\end{eqnarray}
It is easy to see that this is the same as the average number of questions asked in order starting with $n=1$ of the form \lq is the answer $n$?\rq{} it would take to find $X$ .

\subsection*{Q3 - A puzzle which lends itself to information type reasoning}

Suppose that you have $n$ coins, among which there may or may not be
one counterfeit coin. If there is a counterfeit coin it will weigh
either less or more than the other coins. The coins are weighed using
a balance, any number of coins van be put on each side of the balance,
though obviously you will want the same number on each side.
\begin{enumerate}
\item Find an upper bound on the number of coins $n$ so that $k$
  weighings will find the counterfeit coin, if any, and correctly
  declare it to be heavier or lighter.
\item What is the coin-weighing strategy for $k=3$ weighings and 12
  coins,
\end{enumerate}

\subsubsection*{Solution}

So given that one of $n$ coins is counterfeit; there are $2n$
possible configurations, numbering the coins one to $n$, each
possibility is either of the form the $i$th coin is heavier, or the
$i$th coin is lighter. Thus, assuming all possibilities are equally
likely, the random variable $X$ giving the identity and type of the
bent coin has entropy $H(X)=\log{2n}$. What about weighing, $Y$, well
each weighing involves taking two groups of coins and balancing them
and this has three possible outcomes: left heavier, right heavier or
balanced. Obivously, depending on what we have already worked out
about the coins from previous weighings, these possibilities have
different outcomes, for example, at the start, given that one coin is
counterfeit, weighing $n/2$ coins against $n/2$ coins can't give
balanced and the entropy for this measurement will be one
bit. However, we know that $H(Y)<\log{3}$; the most uncertain
measurement is the one where all possibilities are equally likely.

Now, imagine drawing up a weighing statedgy, you are going to do $k$
weighings $Y_1$, $Y_2$ to $Y_k$. The outcome of a weighing is determined by the value of
$x$, the idenity and type of the bent coin, so $H(Y_1,Y_2,\ldots, Y_k|X)=0$. We have
\begin{eqnarray}
  H(X)+H(Y_1,Y_2,\ldots, Y_k|X)&=&H(X,Y_2,\ldots, Y_k)\cr
  &=&H(Y_1,Y_2,\ldots,Y_k)+H(X|Y_1,Y_2,\ldots,Y_k)
\end{eqnarray}
If we have a statedgy that locates and types the counterfeit, there should be no uncertainty in $X$ given the $Y_i$ so 
$H(X|Y_1,Y_2,\ldots,Y_k)=0$. So, if we are able to find and type the counterfeit
\begin{equation}
H(X)=H(Y_1,Y_2,\ldots,Y_k)
\end{equation}
but, from the independence theorem and the bound above
\begin{equation}
H(X)=H(Y_1,Y_2,\ldots,Y_k)\le H(Y_i)\le k\log{3}
\end{equation}
and hence
\begin{equation}
n\le 3^{k}/2
\end{equation}
Hence, if it is possible to identify and type the coin in $k$
weighings, we know we have less than $3^k/2$ coins. This bound may not
be sharp, for particular values of $k$ it may not be possible to
choose a stategy so each $Y$ has $H(Y)=\log{3}$ or so that the entropy
of the joint distribution is equal to the sum of the entropies of the
marginal distribution. However, we do have a bound.

For $k=3$ we have $n\le 13$, in fact, there doesn't seem to be a
solution for $n=13$; there is one for $n=12$. Lets start by numbering
the coins from one to 12. The first weighing is $g_1=\{1,2,3,4\}$
versus $g_2=\{5,6,7,8\}$. If $g_1$ is heavier; then weigh
$g_3=\{1,2,5\}$ versus $g_4=\{3,4,6\}$. Thus $g_3$ and $g_4$ each have
two coins which must be heavier if they are counterfeit and the
remaining two coins, 7 and 8, must be lighter. If $g_3$ is heavier
than $g_4$ this can only be because either 1 or 2 is heavier, or 6 is
lighter; weighing 1 or 2 settles this, if one is heavier than the
other, it is the bent coin, if they balance, 5 is. If $g_3$ and $g_4$
balance then the counterfeit is either 7 or 8 and weighing them gives
the answer. Finally, if $g_1$ and $g_2$ balance the counterfeit coin
must be one of $\{9,10,11,12\}$; start by weighting $g_5=\{9,10\}$
against $g_6=\{11,1\}$: 1 is known not to be counterfiet. If $g_5$ and
$g_6$ balance then the coin can only be 12 and weighing this against 1
gives the answer, otherwise, say $g_5$ is heavier than, either one of
9 and 10 is heavy, or 11 is light, weighing 9 against 10 sorts this
out.




\subsection*{Q4 - Working out entropy and information}

Let $p(x,y)$ be given by $p(0,0)=p(0,1)=p(1,1)=1/3$ and
$p(1,0)=0$. Find $H(X)$, $H(Y)$, $H(X|Y)$, $H(Y|X)$, $H(X,Y)$,
$H(Y)-H(Y|X)$ and $I(X;Y)$.

\subsubsection*{Solution}

So this is just a lot of calculating. First lets work out the two marginal distributions: $p_X(0)=2/3$, $p_X(1)=1/3$, whereas $p_Y(0)=1/3$ and $p_Y(1)=2/3$, hence
\begin{equation}
  H(X)=H(Y)=\frac{1}{3}\log_2{3}+\frac{2}{3}\log_2{\frac{3}{2}}\approx 0.92
\end{equation}
Now for the conditional distributions $p(0|Y=0)=1$, $p(1|Y=0)=0$, but $p(0|Y=1)=p(1|Y=1)=0.5$ and hence
\begin{eqnarray}
  H(X|Y=0)&=&0\cr
  H(X|Y=1)&=&1
\end{eqnarray}
and hence, given $p_Y(1)=2/3$ we have $H(X|Y)=2/3$. $H(Y|X)$ is the same. Now
\begin{equation}
  H(X,Y)=3\frac{1}{3}\log_2(3)\approx 1.58
\end{equation}
Finally we could use $I(X;Y)=H(X)+H(Y)-H(X,Y)\approx 0.25$ or $I(X,Y)=H(X|Y)-H(X)\approx 0.25$ 


\subsection*{Q5 - A question about information in the brain}

Answer just one of these two questions, each is worth equal marks but the
second is much harder than the first, so you'd be better off doing the
first unless you are particularly interested in this topic. Both papers are available in the paper repository in the github.

\begin{enumerate}
\item The original idea of estimating neural information by binning
  spike trains was spread across several papers, but one of the main
  references is \cite{StrongEtAl1998}. One aspect of this paper we
  didn't discuss is the use of extrapolation to estimate the
  information as the number of samples becomes large based on the
  behaviour for smaller numbers of samples. Can you give a short, up
  to five line, summary of what this involves.

\item In \cite{NemenmanEtAl2004} there is a deep commentary on how
  information in neural data is computed. This is a very difficult
  paper and the mathematics towards the end is hard. The aim of this
  question is to read the paper and offer a three or four line overall
  summary of what the paper is trying to do.
\end{enumerate}

\subsubsection*{Solution}

No fixed answer, you need to read the paper!

\bibliographystyle{apa}
\bibliography{../../source/bibliography}{}

\end{document}


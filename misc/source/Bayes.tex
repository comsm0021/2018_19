\documentclass[12pt]{article}
\usepackage{amsfonts, epsfig}
\usepackage[authoryear]{natbib}
\usepackage{graphicx}
\usepackage{fancyhdr}
\pagestyle{fancy}
\lfoot{\texttt{comsm0021.github.io}}
\lhead{Neural Information Processing - Bayes - Conor}
\rhead{\thepage}
\cfoot{}
\begin{document}

\section*{Bayes rule}

Consider $X$ and $Y$, two random variables with $X$ taking values in ${\cal X}$ and $Y$ in ${\cal
  Y}$. The {\sl joint probability} is
\begin{equation}
p_{X,Y}(x,y)=\mbox{probability that $X=x$ and $Y=y$}
\end{equation}
Here is an example with ${\cal X}=\{0,1\}$ and ${\cal Y}=\{0,1,2\}$
\begin{equation}
\begin{array}{l|ll}
&0&1\\
\hline\\[-10pt]
0&1/4&1/8\\
1&1/16&3/8\\
2&1/16&1/8
\end{array}
\end{equation}
So the probability of $(X,Y)=(0,1)$ is $1/16$.

From the joint probability we can define the {\sl marginal distributions}
\begin{eqnarray}
p_{X}(x)&=&\mbox{probability that $X=x$ irrespective of what $Y$ is}\cr
p_{Y}(y)&=&\mbox{probability that $Y=y$ irrespective of what $X$ is}
\end{eqnarray}
and, it follows that
\begin{eqnarray}
p_{X}(x)&=&\sum_{y\in{\cal Y}}p_{X,Y}(x,y)\cr
p_{Y}(y)&=&\sum_{x\in{\cal X}}p_{X,Y}(x,y)
\end{eqnarray}
For the example above
\begin{equation}
\begin{array}{l|ll}
X&0&1\\
\hline\\[-10pt]
 &3/8&5/8
\end{array}
\end{equation}
and
\begin{equation}
\begin{array}{l|lll}
Y&0&1&2\\
\hline\\[-10pt]
&3/8&7/16&3/16
\end{array}
\end{equation}

We can also define the {\sl conditional probabilities}
\begin{eqnarray}
p_{X|Y}(x|y)&=&\mbox{probability that $X=x$ if $Y=y$}\cr
p_{Y|X}(y|x)&=&\mbox{probability that $Y=y$ if $X=x$}
\end{eqnarray}
These are calculated using Bayes rule, basically this says that the probability of $X=x$ and $Y=y$ is the probability of $X=x$ multiplied by the probability of $Y=y$ given that $X=x$:
\begin{equation}
p_{X,Y}(x,y)=p_X(x)p_{Y|X}(y|x)
\end{equation}
and, similarily
\begin{equation}
p_{X,Y}(x,y)=p_Y(y)p_{X|Y}(x|y)
\end{equation}
and, of course, this means
\begin{eqnarray}
p_{Y|X}(y|x)&=&\frac{p_{X,Y}(x,y)}{p_X(x)}\cr
p_{X|Y}(x|y)&=&\frac{p_{X,Y}(x,y)}{p_Y(y)}
\end{eqnarray}
so in the example above
For the example above
\begin{equation}
\begin{array}{l|ll}
X|Y=0&0&1\\
\hline\\[-10pt]
 &2/3&1/3
\end{array}
\end{equation}
\begin{equation}
\begin{array}{l|ll}
X|Y=1&0&1\\
\hline\\[-10pt]
 &1/7&6/7
\end{array}
\end{equation}
\begin{equation}
\begin{array}{l|ll}
X|Y=2&0&1\\
\hline\\[-10pt]
 &1/3&2/3
\end{array}
\end{equation}
and
\begin{equation}
\begin{array}{l|lll}
Y|X=0&0&1&2\\
\hline\\[-10pt]
&4/6&1/6&1/6
\end{array}
\end{equation}
\begin{equation}
\begin{array}{l|lll}
Y|X=1&0&1&2\\
\hline\\[-10pt]
&1/5&3/5&1/5
\end{array}
\end{equation}
Finally, since $p_{X,Y}(x,y)=p_{Y,X}(y,x)$ we can write Bayes rule in the more familiar form
\begin{equation}
p_{Y|X}(y|x)=\frac{p_Y(y)p_{X|Y}(x|y)}{p_X(x)}
\end{equation}



\end{document}

\documentclass{article}

\usepackage{amsmath,amsfonts,amssymb}   

\newif\ifanswer
\answertrue
%\answerfalse

\let\imp=\Rightarrow
\let\iff=\Leftrightarrow
\usepackage{graphicx}
\usepackage{listings}
\usepackage{tikz-qtree}
\lstset{language=C}


\tikzset{every tree node/.style={minimum width=2em,draw,circle},
         blank/.style={draw=none},
         edge from parent/.style=
         {draw,->, edge from parent path={(\tikzparentnode) -- (\tikzchildnode)}},
         level distance=1.5cm}

\begin{document}

\section*{COMS10011 sample paper}

THIS IS STILL A WORK IN PROGRESS! When it is done the following will
apply: This is a sample paper, it has the same style of question as
the real paper, the layout is slightly different to the official exam
layout.

\subsection*{Rubric}{
This paper contains \emph{two} parts. \\
The first section contains \emph {15} short questions.\\ 
Each question is worth \emph{two marks} and all should be attempted.\\
The second section contains \emph {three} long questions.\\
Each long question is worth \emph{20 marks}.\\
The best \emph{two} long question answers will be used for assessment. \\
The maximum for this paper is \emph{70 marks}. \\
Calculators must have the Faculty of Engineering Seal of Approval.}

%MSM: unit specific Latex commands 

\subsection*{Section A: short questions - answer all questions}

\begin{enumerate}

%1 conor
\item What is the definition of Shannon's entropy for a discrete distribution?

  \ifanswer
  For sample space $\mathcal{X}$ we have
  \begin{equation}
    H(X)=-\sum_{x\in\mathcal{X}}p(x)\log{p(x)}
  \end{equation}
  \fi

  %2 conor

\item If you have a finite set of $n$ spike trains and calculate the
  entropy of the spike trains using the discretization method proposed
  by Bialek and co-workers, in general, what limit would the entropy
  reach as the time step is made very small?

  \ifanswer If the time step is small enough each spike train would
  correspond to a different word, with small enough time step, each
  spike in each train would, in general, fall into a different time
  bins. This means there would be $n$ different words, each with one
  occurance and hence probability $1/n$. Thus the entropy would be
  $\log{n}$. I recently had to reject a paper where the authors had
  completely misunderstood this!
  \fi
  
%3 conor
\item If we have a Markov chain $V\rightarrow X\rightarrow H$ what can we say about $p_{V,H|X}(v,h|x)$?

\ifanswer
\begin{equation}
p_{V,H|X}(v,h|x)=p_{V|X}(v|x)p_{H|X}(h|x)
\end{equation}

%4 conor
\item What is the cocktail party problem?

  \ifanswer Often the environment is very noisy with a sound, such as a voice, that is being attended to, not much distinguished in amplitude from other noises. I have read that this situation holds at \lq{}cocktail parties\rq{}, a form of social entertainment. The cocktail party problem is the question of how the brain seperates the sound signal it is attending to from the background noise.
  \fi

%5 conor
\item In the Eriksen flanker task sketch the accuracy versus
reaction time for the consistent, \textbf{HHH}, and inconsistient,
\textbf{HSH}, conditions. The overall scale of the reaction time is
not what is being asked for, rather the shape.

\ifanswer
In both cases the line starts at 0.5 but in the consistient case it
rises quickly to one, in the inconsistient case it falls below 0.5
before rising more slowly to one. [1 mark for starting at 0.5 and rising to 1, the other for the dip in inconsistient].
\fi

%6 conor
\item The $n$-armed bandit task is used in psychological studies of decision making. What is an $n$-armed bandit?

  \ifanswer
  In an $n$-armed bandit the participant has to chose between $n$ options, typically $n$ buttons, each with a different probability of reward.
  \fi

  %7 conor
\item If four options in a decision task have estimated reward values $r_1$, $r_2$, $r_3$ and $r_4$, what is the soft-max probability for chosing the $i$th option?

  \ifanswer
  \begin{equation}
    p_i=\frac{e^{\beta r_i}}{\sum_j e^{\beta r_j}}
  \end{equation}
  for some $\beta$, a parameter determining the exploration to explotation balance.
  \fi
  





  

  
\end{enumerate}
  
\section*{Section B: long questions - answer two questions}

\begin{enumerate}

%question from conor
\item This question is about the Kalman filter. INCOMPLETE
\begin{emumerate}
\item{}[5] Consider two random variables which are conditionally independent with normal distributions:
  \begin{eqnarray}
    p_{X|H}(x|h)&=&\frac{1}{\sqrt{2\pi\sigma_X^2}}e^{-(x-h)^2/2\sigma_X^2}\cr
    p_{Y|H}(y|h)&=&\frac{1}{\sqrt{2\pi\sigma_Y^2}}e^{-(y-h)^2/2\sigma_Y^2}\cr
  \end{equation}
  then $p_{X|H}(x|h)p_{Y|H}(y|h)$ is proportional to a normal distribution in $h$. What is that normal distribution?
 
  \ifanswer  Well this is just an exercise in matching terms, with the obvious notation:
  \begin{equation}
    \frac{1}{\sigma^2}(h^2-2h\mu)=\frac{1}{\sigma_X^2}(h^2-2hx)+\frac{1}{\sigma_Y^2}(h^2-2hy)
  \end{equation}
  so
  \begin{equation}
    \frac{1}{\sigma^2}=\frac{1}{\sigma_X^2}+\frac{1}{\sigma_Y^2}
  \end{equation}
  and
  \begin{equation}
  \mu=\frac{\sigma^2}{\sigma_X^2}x+\frac{\sigma^2}{\sigma_Y^2}y
  \end{equation}

  \end{enumerate}

  
%question from cian

%question split between conor and cian  
\item There are two parts to this question, the first is about information theory, the second is about cian stuff.
  \begin{enumerate}
  \item{}


\begin{enumerate}
\item{}[3] Define $I(X;Y)$ and give a sufficient condition for $I(X;Y)=0$ for non-trivial random variables $X$ and $Y$?
  \ifanswer
  \begin{equation}
  I(X;Y)=\sum_{x,y}p(x,y)\log{\frac{p(x,y)}{p(x)p(y)}}
  \end{equation}
  and this is zero if the argument of the log is zero, that is $p(x,y)=p(x)p(y)$ for all $x$ and $y$, that is when $X$ and $Y$ are independent.
  \fi
\item{}[4]
 Calculate the mutual information between random variables $X$ and $Y$ with sample spaces $\mathcal{X}=\{\mbox{a},\mbox{b},\mbox{c}\}$ and $\mathcal{Y}=\{\alpha,\beta\}$.
  \begin{center}
  \begin{tabular}{c|ccc}
    \hline
    &\mbox{a}&\mbox{b}&\mbox{c}\\
    $\alpha$&0.5&0.125&0\\
    $\beta$&0&0.125&0.25
    \end{tabular}
  \end{center}
You can write the answer in terms of $\log{3}$ and $\log{5}$ if you would prefer.
  
  \ifanswer
So to marginalize we have, using the obvious abuse of notation $p(X)=\{0.5,0.25,0.25\}$ and $p(Y)=\{0.625,0.375\}$ so the table for $p_X(x)p_Y(y)$ is
  \begin{center}
  \begin{tabular}{c|ccc}
    \hline
    &\mbox{a}&\mbox{b}&\mbox{c}\\
    $\alpha$&5/16&5/32&5/32\\
    $\beta$&3/16&3/32&3/32
    \end{tabular}
  \end{center}
  now
  \begin{equation}
    I(X;Y)=\sum_{x,y}p(x,y)\log{\frac{p(x,y)}{p(x)p(y)}}
  \end{equation}
  so
  \begin{eqnarray}
    I(X;Y)&=&\frac{1}{2}\log{\frac{1/2}{5/16}}+\frac{1}{4}\log{\frac{1/8}{5/32}}+\frac{1/4}\log{\frac{1/8}{3/32}}+\frac{1}{4}\log{\frac{1/4}{3/32}}\cr
    &=&\frac{1}{2}\log{\frac{8}{5}}+\frac{1}{4}\log{\frac{4}{5}}+\frac{1/4}\log{\frac{4}{3}}+\frac{1}{4}\log{\frac{8}{3}}\cr
    &=&\frac{13}{4}-\frac{3}{4}\log{5}-\frac{1}{2}\log{3}\approx 0.716
  \end{eqnarray}

  \fi

\item{}[3] Show $I(X;Y)=H(X)-H(X|Y)$.

  \ifanswer
  So we start with the definition of the mutual information
  \begin{equation}
    I(X;Y)=\sum_{x,y} p(x,y)\log{\frac{p(x,y)}{p(x)p(y)}}
  \end{equation}
  and then write $p(x,y)=p(x|y)p(y)$ giving
    \begin{equation}
    I(X;Y)=\sum_{x,y} p(x,y)\log{\frac{p(x|y)}{p(x)}}=\sum_{x,y}p(x,y)\log{p(x|y)}-\sum_{x,y}p(x,y)\log{p(x)}
  \end{equation}
In the first term write $p(x,y)=p(x|y)p(y)$$ again to get the definition of $H(X|Y)$ and in the second term sum over $y$ to marginalize $p(x,y)$ to $p(x)$.
  
\end{enumerate}
\item{}[10] Cian's part
  \end{enumerate}

  
 \end{enumerate}
  
\end{document}
